\chapter{Trabajo futuro}\label{cap:trabajo futuro}

Los objetivos principales de este trabajo han sido completado como se menciona anteriormente. Aún así, se han pensado algunas mejoras que se le podría implementar a esta aplicación.

\begin{enumerate}
    \item \textbf{Usar aplicación completa de forma local:} la aplicación tiene su base de datos de forma \textit{online} y a tiempo real, esto implica a que se necesite internet cada vez que se quiera usar, pero se podría hacer uso de SQL Lite u otra base de datos local para que mantuviera lo existente para usos \textit{offline} y se actualizara una vez se tuviera acceso a internet.
    
    \item \textbf{Gráficas para representar puntuaciones:} las puntuaciones se nos muestran con forma de "bocadillo" mostrándonos la puntuación, la posición y una corona si hemos quedado primero, pero otra forma de mostrar estas puntuaciones, partidas ganadas, perdidas o jugadas en un año, mes o día específico, etc. Sería mediante gráficas. Esto haría que fuera de una forma más visual y sencilla la representación de estos números.
    
    \item \textbf{Más conectividad entre usuarios:} actualmente, la conectividad entre usuarios es mínima. La única conexión que tienen es cuando crean una nueva partida y escriben como participante a otro usuario. Por ello, se podría crear una sección de comentarios de partidas para los usuarios que sean participantes de dichas partidas, valoraciones a las partidas, agregar amigos, ver resultados y partidas de amigos, etc.
    
    \item \textbf{Animaciones y sonidos para la alarma:} en la sección de utilidades tenemos una herramienta que se llama\textit{Cronómetro}, pero no destaca mucho. Para ello, se podría rediseñar con animaciones cuando clicas en alguno de los botones de \textit{play, pausa} o \textit{replay}, o que el circulo que hay alrededor gire, por ejemplo. Además, también se le podrían incluir pitos cada 10 segundos o 1 minuto.
    
    \item \textbf{Lista de juegos expandible por usuarios:} la lista de juegos se obtiene mediante una API, pero siempre puede quedar algún juego fuera de esta API. Por ello, estaría bien implementar una funcionalidad para que los usuarios hagan peticiones para implementar juegos a la lista.
    
    \item \textbf{Subir aplicación a PlayStore:} por último, lo que se podría hacer para mejorar un poco más la aplicación sería dar a los usuarios un método sencillo para poder instalar la aplicación en sus dispositivos sin tener que descargar un archivo .APK. Para ello, la solución más fácil sería subir la aplicación a la PlayStore de Android.
\end{enumerate}