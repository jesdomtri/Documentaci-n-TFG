\chapter{Conclusiones}\label{cap:conclusiones}

A pesar de la situación actual y los problemas que derivan de ella, las conclusiones que saco del proyecto son positivas. Elegí este proyecto porque quería realizar un desarrollo desde cero creando un producto final que fuese usado por un grupo de personas y contribuyese a facilitar el trabajo.

Hablando del desarrollo de la aplicación en general, puedo decir que aun teniendo conocimientos previos sobre aplicaciones Android, se notó la falta de experiencia en el desarrollo de aplicaciones desde cero. Este problema se pudo solucionar mediante la experiencia y el manejo que iba obteniendo mientras progresaba en el proyecto.

Si hablamos sobre los objetivos comentados al principio de este documento, podemos decir que se han cumplido todos los objetivos deseados: 

\begin{enumerate}
    \item Las estadísticas de las partidas en los juegos de mesa son guardadas en una base de datos Firebase completamente en línea.
    \item Estas estadísticas son diferenciadas entre cada participante de dicha partida, es decir, a cada participante se le asigna únicamente sus puntuaciones y las de nadie más.
    \item Al ser una base de datos completamente en línea, el usuario puede obtener todos sus progresos independientemente del dispositivo que esté utilizando.
    \item El usuario tiene a su alcance herramientas para lanzar una gran cantidad y de diferentes tipos de dados, puede contar puntuaciones en directo, hacer uso de los diferentes contadores para juegos de cartas que requieren contadores para múltiples características y poder calcular el tiempo sin la necesidad de estar cambiando de aplicación continuamente.
    \item La aplicación usa la red del dispositivo para obtener y guardar los datos de la base de datos, mientras que puede usar las diferentes herramientas sin tener acceso a internet.
\end{enumerate}

