\chapter{Implementación}\label{cap:implementacion}

A continuación, se comentarán el entorno de programación como aspectos importantes de la implementación.

\section{Entorno de programación}

El lenguaje de programación escogido ha sido Java. Es un lenguaje de programación y una plataforma informática que fue comercializada por primera vez en 1995 por Sun Microsystems. Hay muchas aplicaciones y sitios web que no funcionarán, probablemente, a menos que tengan Java instalado y cada día se crean más. Java es rápido, seguro y fiable.

La principal razón de la elección de este lenguaje es debido a la preferencia del desarrollador, en este caso el alumno, ya que es uno de los lenguajes que maneja con más eficiencia y ha trabajado en proyectos con una temática similar.

Un entorno de desarrollo integrado, llamado también IDE (siglas en inglés de integrated development environment), es un programa informático compuesto por un conjunto de herramientas de programación. Puede dedicarse en exclusiva a un solo lenguaje de programación o bien puede utilizarse para varios.

Un IDE es un entorno de programación que ha sido empaquetado como un programa de aplicación; es decir, que consiste en un editor de código, un compilador, un depurador y un constructor de interfaz gráfica (GUI). Los IDE pueden ser aplicaciones por sí solas o pueden ser parte de aplicaciones existentes.

Los IDE proveen un marco de trabajo amigable para la mayoría de los lenguajes de programación tales como C++, PHP, Python, Java, C\#, Delphi, Visual Basic, Gambas, etc. 

Como IDE para este proyecto utilizamos Android Studio. Es el entorno de desarrollo integrado oficial para la plataforma Android. Está basado en el software IntelliJ IDEA de JetBrains y ha sido publicado de forma gratuita a través de la Licencia Apache 2.0. Ha sido diseñado específicamente para el desarrollo de Android.

\newpage