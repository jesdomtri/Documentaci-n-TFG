\chapter{Plan de pruebas}\label{cap:plan pruebas}

A continuación, presentaré los resultados de las pruebas realizadas en los formatos especificados en el apartado anterior.

\section{Pruebas de código}

\begin{center}
    \rowcolors{1}{mygray}{white}
    \begin{tabulary}{0.8\textwidth}{L|L}
        \textbf{Apartado} & \textbf{Resultado} \\ \hline
        Funcionalidad que revisar & Inicio de conexión con Authentication \\
        Implementado por & FirebaseAuth \\
        Prueba que realizar & Permitir la conexión con Authentication \\
        Resultado válido & Permite la conexión con Authentication \\
        Resultado obtenido & Permite la conexión con Authentication \\
        Prueba satisfactoria & Sí \\
    \end{tabulary} 
\end{center}

\bigskip

\begin{center}
    \rowcolors{1}{mygray}{white}
    \begin{tabulary}{0.8\textwidth}{L|L}
        \textbf{Apartado} & \textbf{Resultado} \\ \hline
        Funcionalidad que revisar & Conexión con la instancia especificada de Authentication \\
        Implementado por & FirebaseAuth\_getInstance \\
        Prueba que realizar & Se conecta con la instancia especificada de Authentication \\
        Resultado válido & Instancia específica de Authentication conectada \\
        Resultado obtenido & Instancia específica de Authentication conectada \\
        Prueba satisfactoria & Sí \\
    \end{tabulary} 
\end{center}

\bigskip

\begin{center}
    \rowcolors{1}{mygray}{white}
    \begin{tabulary}{0.8\textwidth}{L|L}
        \textbf{Apartado} & \textbf{Resultado} \\ \hline
        Funcionalidad que revisar & Inicio de conexión con Realtime Database \\
        Implementado por & FirebaseDatabase \\
        Prueba que realizar & Permitir la conexión con Realtime Database \\
        Resultado válido & Permite la conexión con Realtime Database \\
        Resultado obtenido & Permite la conexión con Realtime Database \\
        Prueba satisfactoria & Sí \\
    \end{tabulary} 
\end{center}

\bigskip

\begin{center}
    \rowcolors{1}{mygray}{white}
    \begin{tabulary}{0.8\textwidth}{L|L}
        \textbf{Apartado} & \textbf{Resultado} \\ \hline
        Funcionalidad que revisar & Conexión con la instancia especificada de Realtime Database \\
        Implementado por & FirebaseDatabase\_getInstance \\
        Prueba que realizar & Se conecta con la instancia especificada de Realtime Database \\
        Resultado válido & Instancia específica de Realtime Database conectada \\
        Resultado obtenido & Instancia específica de Realtime Database conectada \\
        Prueba satisfactoria & Sí \\
    \end{tabulary} 
\end{center}

\bigskip

\begin{center}
    \rowcolors{1}{mygray}{white}
    \begin{tabulary}{0.8\textwidth}{L|L}
        \textbf{Apartado} & \textbf{Resultado} \\ \hline
        Funcionalidad que revisar & Inicio de conexión con las referencias en Realtime Database \\
        Implementado por & DatabaseReference \\
        Prueba que realizar & Permitir la conexión con las referencias en Realtime Database \\
        Resultado válido & Permite la conexión con las referencias en Realtime Database \\
        Resultado obtenido & Permite la conexión con las referencias en Realtime Database \\
        Prueba satisfactoria & Sí \\
    \end{tabulary} 
\end{center}

\bigskip

\begin{center}
    \rowcolors{1}{mygray}{white}
    \begin{tabulary}{0.8\textwidth}{L|L}
        \textbf{Apartado} & \textbf{Resultado} \\ \hline
        Funcionalidad que revisar & Conexión con la Referencia especificada de Realtime Database \\
        Implementado por & DatabaseReference\_getReference \\
        Prueba que realizar & Se conecta con la instancia especificada de Realtime Database \\
        Resultado válido & Referencia específica de Realtime Database conectada \\
        Resultado obtenido & Referencia específica de Realtime Database conectada \\
        Prueba satisfactoria & Sí \\
    \end{tabulary} 
\end{center}

\bigskip

\begin{center}
    \rowcolors{1}{mygray}{white}
    \begin{tabulary}{0.8\textwidth}{L|L}
        \textbf{Apartado} & \textbf{Resultado} \\ \hline
        Funcionalidad que revisar & Mantener conectado al usuario aún cerrando la aplicación \\
        Implementado por & SharedPreferences \\
        Prueba que realizar & Mantener iniciada la sesión \\
        Resultado válido & Inicio de sesión mantenido aunque se cierre la aplicación \\
        Resultado obtenido & Inicio de sesión mantenido aunque se cierre la aplicación \\
        Prueba satisfactoria & Sí \\
    \end{tabulary}
\end{center}

\bigskip

\begin{center}
    \rowcolors{1}{mygray}{white}
    \begin{tabulary}{0.8\textwidth}{L|L}
        \textbf{Apartado} & \textbf{Resultado} \\ \hline
        Funcionalidad que revisar & Edición de las preferencias del SharedPreferences \\
        Implementado por & SharedPreferences\_Editor \\
        Prueba que realizar & Editar las preferencias del SharedPreferences para guardar o no la sesión actual \\
        Resultado válido & Preferencias del SharedPreferences editadas \\
        Resultado obtenido & Preferencias del SharedPreferences editadas \\
        Prueba satisfactoria & Sí \\
    \end{tabulary}
\end{center}

\bigskip

\begin{center}
    \rowcolors{1}{mygray}{white}
    \begin{tabulary}{0.8\textwidth}{L|L}
        \textbf{Apartado} & \textbf{Resultado} \\ \hline
        Funcionalidad que revisar & Modificación de características de las ventanas \\
        Implementado por & requestWindowFeature \\
        Prueba que realizar & Modificar las características de las ventanas \\
        Resultado válido & Características de las ventanas modificada \\
        Resultado obtenido & Características de las ventanas modificada \\
        Prueba satisfactoria & Sí \\
    \end{tabulary}
\end{center}

\bigskip

\begin{center}
    \rowcolors{1}{mygray}{white}
    \begin{tabulary}{0.8\textwidth}{L|L}
        \textbf{Apartado} & \textbf{Resultado} \\ \hline
        Funcionalidad que revisar & Configuración de la barra de la aplicación \\
        Implementado por & getSupportActionBar \\
        Prueba que realizar & Configurar la barra de la aplicación \\
        Resultado válido & Barra de la aplicación configurada \\
        Resultado obtenido & Barra de la aplicación configurada \\
        Prueba satisfactoria & Sí \\
    \end{tabulary}
\end{center}

\bigskip

\begin{center}
    \rowcolors{1}{mygray}{white}
    \begin{tabulary}{0.8\textwidth}{L|L}
        \textbf{Apartado} & \textbf{Resultado} \\ \hline
        Funcionalidad que revisar & Asignación de vistas a las actividades Java \\
        Implementado por & setContentView \\
        Prueba que realizar & Asignar las vistas a las actividades Java \\
        Resultado válido & Vistas a las actividades Java asignadas \\
        Resultado obtenido & Vistas a las actividades Java asignadas \\
        Prueba satisfactoria & Sí \\
    \end{tabulary}
\end{center}

\bigskip

\begin{center}
    \rowcolors{1}{mygray}{white}
    \begin{tabulary}{0.8\textwidth}{L|L}
        \textbf{Apartado} & \textbf{Resultado} \\ \hline
        Funcionalidad que revisar & Creación de una \textit{intención} para iniciar otra actividad \\
        Implementado por & Intent \\
        Prueba que realizar & Crear una \textit{intención} para iniciar otra actividad \\
        Resultado válido & \textit{Intención} para iniciar otra actividad creada \\
        Resultado obtenido & \textit{Intención} para iniciar otra actividad creada \\
        Prueba satisfactoria & Sí \\
    \end{tabulary}
\end{center}

\bigskip

\begin{center}
    \rowcolors{1}{mygray}{white}
    \begin{tabulary}{0.8\textwidth}{L|L}
        \textbf{Apartado} & \textbf{Resultado} \\ \hline
        Funcionalidad que revisar & Adición de claves nuevas en la \textit{intención} que inicia una actividad \\
        Implementado por & Intent\_putExtra \\
        Prueba que realizar & Añadir claves nuevas en la \textit{intención} que inicia una actividad \\
        Resultado válido & Claves nuevas en la \textit{intención} que inicia una actividad añadidas \\
        Resultado obtenido & Claves nuevas en la \textit{intención} que inicia una actividad añadidas \\
        Prueba satisfactoria & Sí \\
    \end{tabulary}
\end{center}

\bigskip

\begin{center}
    \rowcolors{1}{mygray}{white}
    \begin{tabulary}{0.8\textwidth}{L|L}
        \textbf{Apartado} & \textbf{Resultado} \\ \hline
        Funcionalidad que revisar & Devolución de la \textit{intención} que inicia una actividad \\
        Implementado por & getIntent \\
        Prueba que realizar & Devolver la \textit{intención} que inicia una actividad \\
        Resultado válido & \textit{Intención} que inicia una actividad devuelta \\
        Resultado obtenido & \textit{Intención} que inicia una actividad devuelta \\
        Prueba satisfactoria & Sí \\
    \end{tabulary}
\end{center}

\bigskip

\begin{center}
    \rowcolors{1}{mygray}{white}
    \begin{tabulary}{0.8\textwidth}{L|L}
        \textbf{Apartado} & \textbf{Resultado} \\ \hline
        Funcionalidad que revisar & Obtención de claves en la \textit{intención} que inicia la actividad actual \\
        Implementado por & getIntent\_getStringExtra \\
        Prueba que realizar & Obtener claves de la \textit{intención} que inicia la actividad actual \\
        Resultado válido & Claves de la \textit{intención} que inicia la actividad actual obtenidas \\
        Resultado obtenido & Claves de la \textit{intención} que inicia la actividad actual obtenidas \\
        Prueba satisfactoria & Sí \\
    \end{tabulary}
\end{center}

\bigskip

\begin{center}
    \rowcolors{1}{mygray}{white}
    \begin{tabulary}{0.8\textwidth}{L|L}
        \textbf{Apartado} & \textbf{Resultado} \\ \hline
        Funcionalidad que revisar & Búsqueda de una vista por su identificador \\
        Implementado por & findViewById \\
        Prueba que realizar & Buscar una vista por su identificador \\
        Resultado válido & Vista encontrada por su identificador \\
        Resultado obtenido & Vista encontrada por su identificador \\
        Prueba satisfactoria & Sí \\
    \end{tabulary}
\end{center}

\bigskip

\begin{center}
    \rowcolors{1}{mygray}{white}
    \begin{tabulary}{0.8\textwidth}{L|L}
        \textbf{Apartado} & \textbf{Resultado} \\ \hline
        Funcionalidad que revisar & Realización de tareas si se ejecuta la anterior exitosamente \\
        Implementado por & addOnSuccessListener \\
        Prueba que realizar & Realizar tareas si se ejecuta la anterior exitosamente \\
        Resultado válido & Tareas realizadas si la anterior fue exitosa \\
        Resultado obtenido & Tareas realizadas si la anterior fue exitosa \\
        Prueba satisfactoria & Sí \\
    \end{tabulary}
\end{center}

\bigskip

\begin{center}
    \rowcolors{1}{mygray}{white}
    \begin{tabulary}{0.8\textwidth}{L|L}
        \textbf{Apartado} & \textbf{Resultado} \\ \hline
        Funcionalidad que revisar & Realización de tareas si se completó la anterior \\
        Implementado por & addOnCompleteListener \\
        Prueba que realizar & Realizar tareas si se completó la anterior \\
        Resultado válido & Tareas realizadas si la anterior fue completada \\
        Resultado obtenido & Tareas realizadas si la anterior fue completada \\
        Prueba satisfactoria & Sí \\
    \end{tabulary}
\end{center}

\bigskip

\begin{center}
    \rowcolors{1}{mygray}{white}
    \begin{tabulary}{0.8\textwidth}{L|L}
        \textbf{Apartado} & \textbf{Resultado} \\ \hline
        Funcionalidad que revisar & Realización de tareas si ha habido cambios \\
        Implementado por & setOnCheckedChangeListener \\
        Prueba que realizar & Realizar tareas si ha habido cambios \\
        Resultado válido & Tareas realizadas si ha habido cambios \\
        Resultado obtenido & Tareas realizadas si ha habido cambios \\
        Prueba satisfactoria & Sí \\
    \end{tabulary}
\end{center}

\bigskip

\begin{center}
    \rowcolors{1}{mygray}{white}
    \begin{tabulary}{0.8\textwidth}{L|L}
        \textbf{Apartado} & \textbf{Resultado} \\ \hline
        Funcionalidad que revisar & Realización de tareas si el objeto ha sido clicado \\
        Implementado por & setOnClickListener \\
        Prueba que realizar & Realizar tareas si el objeto ha sido clicado \\
        Resultado válido & Tareas realizadas si el objeto ha sido clicado \\
        Resultado obtenido & Tareas realizadas si el objeto ha sido clicado \\
        Prueba satisfactoria & Sí \\
    \end{tabulary}
\end{center}

\bigskip

\begin{center}
    \rowcolors{1}{mygray}{white}
    \begin{tabulary}{0.8\textwidth}{L|L}
        \textbf{Apartado} & \textbf{Resultado} \\ \hline
        Funcionalidad que revisar & Realización de tareas si se escribe alguna consulta \\
        Implementado por & setOnQueryTextListener \\
        Prueba que realizar & Realizar tareas si se escribe alguna consulta \\
        Resultado válido & Tareas realizadas si se escribe alguna consulta \\
        Resultado obtenido & Tareas realizadas si se escribe alguna consulta \\
        Prueba satisfactoria & Sí \\
    \end{tabulary}
\end{center}

\bigskip

\begin{center}
    \rowcolors{1}{mygray}{white}
    \begin{tabulary}{0.8\textwidth}{L|L}
        \textbf{Apartado} & \textbf{Resultado} \\ \hline
        Funcionalidad que revisar & Creación de una barra de navegación \\
        Implementado por & BottomNavigationView \\
        Prueba que realizar & Crear una barra de navegación \\
        Resultado válido & Barra de navegación creada \\
        Resultado obtenido & Barra de navegación creada \\
        Prueba satisfactoria & Sí \\
    \end{tabulary}
\end{center}

\bigskip

\begin{center}
    \rowcolors{1}{mygray}{white}
    \begin{tabulary}{0.8\textwidth}{L|L}
        \textbf{Apartado} & \textbf{Resultado} \\ \hline
        Funcionalidad que revisar & Cambio de vista clicando sobre la barra de navegación \\
        Implementado por & BottomNavigationView\_OnNavigationItemSelectedListener \\
        Prueba que realizar & Cambiar de vista clicando sobre la barra de navegación \\
        Resultado válido & Vista cambiada clicando sobre la barra de navegación \\
        Resultado obtenido & Vista cambiada clicando sobre la barra de navegación \\
        Prueba satisfactoria & Sí \\
    \end{tabulary}
\end{center}

\bigskip

\begin{center}
    \rowcolors{1}{mygray}{white}
    \begin{tabulary}{0.8\textwidth}{L|L}
        \textbf{Apartado} & \textbf{Resultado} \\ \hline
        Funcionalidad que revisar & Mostración de mensajes hacia el usuario \\
        Implementado por & Toast \\
        Prueba que realizar & Mostrar mensajes hacia el usuario \\
        Resultado válido & Mensajes hacia el usuario mostrados \\
        Resultado obtenido & Mensajes hacia el usuario mostrados \\
        Prueba satisfactoria & Sí \\
    \end{tabulary}
\end{center}

\newpage