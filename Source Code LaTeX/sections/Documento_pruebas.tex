\chapter{Documento de pruebas}\label{cap:documento pruebas}

Las pruebas de software son las investigaciones empíricas y técnicas cuyo objetivo es proporcionar información objetiva e independiente sobre la calidad del producto a la parte interesada.

Las pruebas son básicamente un conjunto de actividades dentro del desarrollo de software. Dependiendo del tipo de pruebas, estas actividades podrán ser implementadas en cualquier momento de dicho proceso de desarrollo.

\section{Formato de las pruebas realizadas}

Estas pruebas constituyen la comprobación de que el código implementado se comporta como debe, sin errores en el código ni problemas en el mismo. También comprenden las pruebas de la integridad de los datos, como puede ser la recuperación de estos y el manejo del acceso con datos incorrectos.

Para el desarrollo de las pruebas seguiré un esquema de formularios donde plantearemos:

\begin{itemize}
    \item La funcionalidad que revisar.
    \item Parte de código o nombre que implementa dicha funcionalidad.
    \item Prueba que realizar.
    \item El resultado que daremos por válido.
    \item El resultado obtenido tras una prueba.
    \item Si la prueba se ha pasado satisfactoriamente.
\end{itemize}

\bigskip

El formato de esta información se presentará en una tabla como la que sigue:

\bigskip

\begin{center}
    \rowcolors{1}{mygray}{white}
    \begin{tabulary}{0.8\textwidth}{L|L}
        \textbf{Apartado} & \textbf{Resultado} \\ \hline
        Funcionalidad que revisar & \hspace{5cm} \\
        Implementado por & \hspace{5cm} \\
        Prueba que realizar & \hspace{5cm} \\
        Resultado válido & \hspace{5cm} \\
        Resultado obtenido & \hspace{5cm} \\
        Prueba satisfactoria & \hspace{5cm} \\
    \end{tabulary} 
\end{center}

\newpage